% !TeX root = A6.tex
\documentclass{article}
\usepackage[utf8]{inputenc}
\usepackage{amsmath}  
\usepackage[dvipdfmx]{graphicx}
\usepackage{caption}
\usepackage{float}
\usepackage{url}
\usepackage[dvipdfmx, x11names]{xcolor}

\begin{document}

\section{1週目 - 実験1}
\subsection{実験結果}
\subsection{課題1}

\section{1週目 - 実験2}
\subsection{実験結果}
\subsection{課題2}
 以下の式を用いて曲線フィッティングを行った.
\begin{equation}
K = aV^b
\end{equation}
\noindent
ただし、VはUpper Level と Lower Level の中間の値である. 両辺の対数を取ると,
\begin{equation}
\log K = \log a + b \log V
\end{equation}
\noindent
となる。これを最小二乗法で解くと,
\begin{equation}
a = \exp(\log a), b = b
\end{equation}
となる.
\noindent
計算結果は, $a = 3.87 \times 10^8$, $b = -3.80$であった.
作成したグラフを以下に図2-1として示す.


\section{1週目 - 実験3}
\subsection{実験結果}
\subsection{課題3}

\section{2週目 - 実験}
\subsection{実験結果}
\subsection{課題}


\end{document}