% !TeX root = A6.tex
\documentclass{article}
\usepackage[utf8]{inputenc}
\usepackage{amsmath}  
\usepackage[dvipdfmx]{graphicx}
\usepackage{caption}
\usepackage{float}
\usepackage{url}
\usepackage[dvipdfmx, x11names]{xcolor}

\begin{document}

\section{1週目 - 実験1}
\subsection{課題1}
 以下に5つのデータの蛍光量と, その平均値と標準偏差を示す.
蛍光量は, 抵抗値を$R = 50 \Omega$, 増幅率を$A = 10^6 $, として, 得られた電圧から以下の式で求めた.

\begin{equation}
    \text{蛍光量} = \frac{1}{AR} \int_0^T |V(t)| dt
\end{equation}

ただし, $T$は時間である.

%表
\begin{table}[h]
    \centering
    \captionsetup{labelformat=empty}
    \caption{表1-1: 蛍光量のデータ}
    \begin{tabular}{|c|c|c|}
        \hline
        データ名 & 蛍光量 \\ \hline
        データ01 & $2.13 \times 10^{-17}$ \\ \hline
        データ02 & $2.58 \times 10^{-17}$ \\ \hline
        データ03 & $1.95 \times 10^{-17}$ \\ \hline
        データ04 & $2.65 \times 10^{-17}$ \\ \hline
        データ05 & $1.94 \times 10^{-17}$ \\ \hline
        平均値 & $2.25 \times 10^{-17}$ \\ \hline
        標準偏差 & $3.07 \times 10^{-18}$ \\ \hline
    \end{tabular}
\end{table}


\section{1週目 - 実験2}
\subsection{課題2}
 実験で得られたデータを以下に表2-1として示す.
\begin{figure}[H]
    \centering
    \captionsetup{labelformat=empty}
    \caption{表2-1: 宇宙線のエネルギースペクトル}
    \includegraphics[width=10cm]{Pictures/data1-2-1.png}
\end{figure}

\noindent
また, 以下の式を用いて曲線フィッティングを行った.
\begin{equation}
K = aV^b
\end{equation}
\noindent
ただし、VはUpper Level と Lower Level の中間の値である. 両辺の対数を取ると,

\begin{equation}
\log K = \log a + b \log V
\end{equation}

\noindent
となる。これを最小二乗法で解くと,
\begin{equation}
a = \exp(\log a), b = b
\end{equation}
となる.
\noindent
計算結果は, $a = 3.87 \times 10^8$, $b = -3.80$であった.
作成したグラフを以下に図2-1として示す.

\begin{figure}[H]
    \centering
    \captionsetup{labelformat=empty}
    \includegraphics[width=10cm]{Pictures/fig1-2.png}
    \caption{図2-1: 宇宙線のエネルギースペクトル}
\end{figure}


\section{1週目 - 実験3}
\subsection{課題3-1}
 実験により得たデータ100個の$\Delta t$を用いて入射角$\theta$を以下の式で求めた.
\begin{equation}
    \theta = sin^{-1}\left(\frac{2.5}{c \Delta t}\right)
\end{equation}
\noindent
宇宙線の入射角度の分布をヒストグラムとして, 以下に図3-1として図示する. ただし, %
無効なデータ数が5件存在し, 削除している.


\begin{figure}[H]
    \centering
    \captionsetup{labelformat=empty}
    \includegraphics[width=10cm]{Pictures/fig1-3.png}
    \caption{図3-1: 宇宙線の入射角度の分布}
\end{figure}

\subsection{課題3-2}
医療分野において, 体内のがん細胞などを画像化するPET検査でコインシデンス計測が使用される.%
片方の検出器だけに当たった迷走放射線や, 宇宙線などのノイズを自動的に排除できるため, 極めてクリアな画像が得られる.




\section{2週目 - 実験}
\subsection{課題2-1}
以下に実験で得られたスペクトルデータ(22 ~ 26)を図4-1から図4-5として示す.

\begin{figure}[H]
    \centering
    \captionsetup{labelformat=empty}
    \includegraphics[width=10cm]{Pictures/A6-2_1.png}
    \caption{図4-1: 1つ目のデータ}
\end{figure}

\begin{figure}[H]
    \centering
    \captionsetup{labelformat=empty}
    \includegraphics[width=10cm]{Pictures/A6-2_2.png}
    \caption{図4-2: 2つ目のデータ}
\end{figure}

\begin{figure}[H]
    \centering
    \captionsetup{labelformat=empty}
    \includegraphics[width=10cm]{Pictures/A6-2_3.png}
    \caption{図4-3: 3つ目のデータ}
\end{figure}

\begin{figure}[H]
    \centering
    \captionsetup{labelformat=empty}
    \includegraphics[width=10cm]{Pictures/A6-2_4.png}
    \caption{図4-4: 4つ目のデータ}
\end{figure}

\begin{figure}[H]
    \centering
    \captionsetup{labelformat=empty}
    \includegraphics[width=10cm]{Pictures/A6-2_5.png}
    \caption{図4-5: 5つ目のデータ}
\end{figure}

\noindent
実験と実験で得られたスペクトルデータとNISTSpectraldataの理論スペクトルデータと比較して%
ターゲットの種類(Cu, Fe, Si, C)をそれぞれ特定する.%
図4-1から4-5のそれぞれに対し, 4つ物質の理論値のデータのそれぞれを重ねて比較したグラフを%
以下に図4-6から図4-10として示す. ただし, Akiは計測データの最大値に合わせてスケーリング%
したものであり, 強度そのものではない.

\begin{figure}[H]
    \centering
    \captionsetup{labelformat=empty}
    \includegraphics[width=10cm]{Pictures/A6-2_6.png}
    \caption{図4-6: 1つ目のデータと理論値の比較}
\end{figure}

\begin{figure}[H]
    \centering
    \captionsetup{labelformat=empty}
    \includegraphics[width=10cm]{Pictures/A6-2_7.png}
    \caption{図4-7: 2つ目のデータと理論値の比較}
\end{figure}

\begin{figure}[H]
    \centering
    \captionsetup{labelformat=empty}
    \includegraphics[width=10cm]{Pictures/A6-2_8.png}
    \caption{図4-8: 3つ目のデータと理論値の比較}
\end{figure}

\begin{figure}[H]
    \centering
    \captionsetup{labelformat=empty}
    \includegraphics[width=10cm]{Pictures/A6-2_9.png}
    \caption{図4-9: 4つ目のデータと理論値の比較}
\end{figure}

\begin{figure}[H]
    \centering
    \captionsetup{labelformat=empty}
    \includegraphics[width=10cm]{Pictures/A6-2_10.png}
    \caption{図4-10: 1つ目のデータと理論値の比較}
\end{figure}

\noindent
結果を以下にまとめる.
\begin{itemize}
    \item 図4-6より, 実験データとFeのデータが500nm以下ではピークが多数存在し,% 
    発振波長が多いという特徴が一致している.
    \item 図4-7より, 実験データとCのデータが各ピークにおいて発振波長が一致している.
    \item 図4-8より, 実験データとSiのデータが500nm付近, 550 ~ 600nm付近で発振の%
    特徴が一致している.
    \item 図4-9より, 実験データとCuのデータが400 ~ 525nm付近での発振が一致していることが分かる.
    \item 図4-10より, 実験データとCのデータが500nm付近, 550 ~ 600nm付近で発振の%
    特徴が一致している.
\end{itemize}




\subsection{課題2-2}
 複数の方向から計測するコインシデンス計測では, 粒子や光子が発生した位置の特定や, 角度分布の測定が%
可能になる. また, 複数方向からの測定によりノイズ除去能力が高まる.

\section{参考文献}

\begin{itemize}
    \item コインシデンス計測の応用例\\
    \url{https://www.rinshokaku.com/contents/pdf/sec7/6.pdf}
    \item 複数の方向から計測するコインシデンス計測の効果\\
    \url{http://www-atom.mls.eng.osaka-u.ac.jp/jp/research/research_4-1.html}  
\end{itemize}


\end{document}