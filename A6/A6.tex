% !TeX root = A6.tex
\documentclass{article}
\usepackage[utf8]{inputenc}
\usepackage{amsmath}  
\usepackage[dvipdfmx]{graphicx}
\usepackage{caption}
\usepackage{float}
\usepackage{url}
\usepackage[dvipdfmx, x11names]{xcolor}

\begin{document}

\section{1週目 - 実験1}
\subsection{課題1}
 以下に5つのデータの蛍光量と, その平均値と標準偏差を示す.
蛍光量は, 抵抗値を$R = 50 \Omega$, 増幅率を$A = 10^6 $, として, 得られた電圧から以下の式で求めた.

\begin{equation}
    \text{蛍光量} = \frac{1}{AR} \int_0^T |V(t)| dt
\end{equation}

ただし, $T$は時間である.

%表
\begin{table}[h]
    \centering
    \captionsetup{labelformat=empty}
    \caption{表1-1: 蛍光量のデータ}
    \begin{tabular}{|c|c|c|}
        \hline
        データ名 & 蛍光量 \\ \hline
        データ01 & $2.13 \times 10^{-17}$ \\ \hline
        データ02 & $2.58 \times 10^{-17}$ \\ \hline
        データ03 & $1.95 \times 10^{-17}$ \\ \hline
        データ04 & $2.65 \times 10^{-17}$ \\ \hline
        データ05 & $1.94 \times 10^{-17}$ \\ \hline
        平均値 & $2.25 \times 10^{-17}$ \\ \hline
        標準偏差 & $3.07 \times 10^{-18}$ \\ \hline
    \end{tabular}
\end{table}


\section{1週目 - 実験2}
\subsection{課題2}
 実験で得られたデータを以下に表2-1として示す.

以下の式を用いて曲線フィッティングを行った.
\begin{equation}
K = aV^b
\end{equation}
\noindent
ただし、VはUpper Level と Lower Level の中間の値である. 両辺の対数を取ると,
\begin{equation}
\log K = \log a + b \log V
\end{equation}
\noindent
となる。これを最小二乗法で解くと,
\begin{equation}
a = \exp(\log a), b = b
\end{equation}
となる.
\noindent
計算結果は, $a = 3.87 \times 10^8$, $b = -3.80$であった.
作成したグラフを以下に図2-1として示す.

\begin{figure}[H]
    \centering
    \captionsetup{labelformat=empty}
    \includegraphics[width=10cm]{Pictures/fig1-2.png}
    \caption{図2-1: 宇宙線のエネルギースペクトル}
\end{figure}


\section{1週目 - 実験3}
\subsection{課題3}
 実験により得たデータ100個の$\Delta t$を用いて入射角$\theta$を以下の式で求めた.
\begin{equation}
    \theta = sin^{-1}\left(\frac{2.5}{c \Delta t}\right)
\end{equation}
\noindent
宇宙線の入射角度の分布をヒストグラムとして, 以下に図3-1として図示する.

\begin{figure}[H]
    \centering
    \captionsetup{labelformat=empty}
    \includegraphics[width=10cm]{Pictures/fig1-3.png}
    \caption{図3-1: 宇宙線の入射角度の分布}
\end{figure}


\section{2週目 - 実験}
\subsection{課題}


\end{document}