% !TeX root = A5_2.tex
\documentclass{article}
\usepackage[utf8]{inputenc}
\usepackage{amsmath}  
\usepackage[dvipdfmx]{graphicx}
\usepackage{caption}
\usepackage{float}
\usepackage{url}
\usepackage[dvipdfmx, x11names]{xcolor}
\usepackage{listings}
\usepackage{amsmath}

\begin{document}
\section{目的}
 気体放電の物理的機構について学習し, 生成されるプラズマの性質を静電プローブ測定を%
通じて理解することが本実験の目的である.

\section{実験結果}
\subsection{実験 I}
 実験Iで得られた圧力$P [Pa]$と電圧$V[V]$および、それらの値から計算した%
$\Phi, \gamma$を極板間距離別に%
それぞれ, 表1-1, 1-2, 1-3, 1-4として以下に示す.

\begin{table}[H]
    \centering
    \captionsetup{labelformat=empty}
    \caption{表1-1: 圧力-電極間距離と放電開始電圧の関係(d = 0.5 cm)}
    \label{tab:data1}
    \begin{tabular}{|c|c|c|c|}
    \hline
    $P$ [Pa] & $V$ [V] & $\Phi$ & $\gamma$ \\
    \hline
    40 & 541 & 5.00 & 0.00676 \\
    \hline
    60 & 425 & 5.69  & 0.00338 \\
    \hline
    80 & 386 & 5.79 & 0.00308 \\
    \hline
    100 & 370 & 5.58 & 0.00378 \\
    \hline
    200 & 450 & 5.46 & 0.00429\\
    \hline
    \end{tabular}
\end{table}

\begin{table}[H]
    \centering
    \captionsetup{labelformat=empty}
    \caption{表1-2: 圧力-電極間距離と放電開始電圧の関係(d = 1.0 cm)}
    \begin{tabular}{|c|c|c|c|}
    \hline
    $P$ [Pa] & $V$ [V] & $\Phi$ & $\gamma$  \\
    \hline 
    40 & 445 & 3.53 & 0.0301 \\
    \hline
    60 & 405 & 4.52 & 0.0110 \\
    \hline
    80 & 362 & 4.93 & 0.00725 \\
    \hline
    100 & 390 & 5.60 & 0.00372 \\
    \hline
    200 & 445 & 6.60 & 0.00137 \\
    \hline
    \end{tabular}
\end{table}

\begin{table}[H]
    \centering
    \captionsetup{labelformat=empty}
    \caption{表1-3: 圧力-電極間距離と放電開始電圧の関係(d = 1.5 cm)}
    \begin{tabular}{|c|c|c|c|}
    \hline
    $P$ [Pa] & $V$ [V] & $\Phi$ & $\gamma$ \\
    \hline
    40 & 415 & 1.98 & 0.160\\
    \hline
    60 & 362 & 2.70 & 0.0719\\
    \hline
    80 & 379 & 3.38 & 0.0351\\
    \hline
    100 & 399 & 4.01 & 0.0185\\
    \hline
    200 & 509 & 6.60 & 0.00137\\
    \hline
    \end{tabular}
\end{table}

\begin{table}[H]
    \centering
    \captionsetup{labelformat=empty}
    \caption{表1-4: 圧力-電極間距離と放電開始電圧の関係(d = 2.0 cm)}
    \begin{tabular}{|c|c|c|c|}
    \hline
    $P$ [Pa] & $V$ [V] & $\Phi$ & $\gamma$ \\
    \hline 
    40 & 397 & 3.53 & 0.00552\\
    \hline
    60 & 384 & 4.52 & 0.00316\\
    \hline
    80 & 417 & 4.93 & 0.00180\\
    \hline
    100 & 473 & 5.60 & 0.000830\\
    \hline
    200 & 603 & 6.60 & 0.000649\\
    \hline
    \end{tabular}
\end{table}

\noindent
ただし, $\Phi, \gamma$は以下の式(1), (2)より導出した.%
定数については, 
\\
$A = 11.3 m^{-1}Pa^{-1}, B = 274 V/(mPa)$とした.

\begin{equation}
    \Phi = \alpha d = Apdexp(\frac{Bpd}{V})
\end{equation}

\begin{equation}
    \gamma = \frac{1}{exp(\Phi) - 1}
\end{equation}

\noindent
また, Vの理論式を以下の式(3)で求めた.

\begin{equation}
    V = \frac{Bpd}{ln(\frac{Apd}{\Phi})} = \frac{Bpd}{ln(\frac{Apd}{ln(1 + 1 / \gamma)})}
\end{equation}

\noindent
計算時には$\gamma$は表1-1から1-4の$\gamma$の平均値$\gamma_{ave} = 0.01$を使用した.
以下に理論式と計測値を比較したグラフを以下に図1-1として示す.

\begin{figure}[H]
    \centering
    \captionsetup{labelformat=empty}
    \includegraphics[width=10cm]{Pictures/A5-2-1.png}
    \caption{図1-1: 圧力-電極間距離と放電開始電圧の関係}
\end{figure}


\subsection{実験 II の結果}
 以下に実験IIで得られたプローブ電圧$V_p [V]$と電流$I_p [mA]$のデータを以下の表2-1に示す.

\begin{table}[H]
    \centering
    \captionsetup{labelformat=empty}
    \caption{表2-1: プローブ電流-電圧特性データ}
    \begin{tabular}{|c|c|}
    \hline
    $V_p$ [V] & $I_p$ [mA] \\
    \hline
    -60 & -0.0033 \\
    \hline
    -50 & -0.0027 \\
    \hline
    -40 & -0.0020 \\
    \hline
    -30 & -0.0012 \\
    \hline
    -20 & 0.0000 \\
    \hline
    -10 & 0.0330 \\
    \hline
    0   & 0.1320 \\
    \hline
    10  & 0.4630 \\
    \hline
    20  & 4.6600 \\
    \hline
    30  & 5.2200 \\
    \hline
    40  & 6.6900 \\
    \hline
    50  & 6.9200 \\
    \hline
    60  & 7.1500 \\
    \hline
    \end{tabular}
\end{table}

また, 表2-1のデータをグラフにしたものを図2-1として示す.
\begin{figure}[H]
    \centering
    \captionsetup{labelformat=empty}
    \includegraphics[width=10cm]{Pictures/A5-2-2.png}
    \caption{図2-1: プローブ電流-電圧特性}
\end{figure}

実験IIで得られたデータから, 電子温度$T_e [eV]$とプラズマ電位$V_s [V]$を求める.
ここでは, ボルツマン定数$k_B = 1.38 \times 10^{-23} [J/K]$, 電子の電荷$e = 1.602 \times 10^{-19} [C]$, %
電子の質量$m = 9.109 \times 10^{-31} [kg]$, プローブの表面積$S = 45.36 [cm^2]$とした.

図2-1において, 電子飽和電流は$V_p = 20 [V]$の時$I_{es} = 4.66 [mA]$とする.
$V_p > V_s$の時, 電子飽和電流は以下の式(4)で表される.

\begin{equation}
    I_{es} =  Sen_e (\frac{k_B T_e}{2 \pi m_e})^{1/2}
\end{equation}

\noindent
よって, $n_e$は, 以下の式(5)により求まる.

\begin{equation}
    n_e = I_{es} / Se (\frac{k_B T_e}{2 \pi m_e})^{1/2}
\end{equation}

\noindent
また, $V_p$を徐々に下げていくと, 電子電流$I_e$はプローブ電流$I_p$とイオン電流$I_i$の%
和を取り, 以下の式で表される.

\begin{equation}
    I_e = I_p + I_i =  I_{es} exp(\frac{e(V_p - V_s)}{k_B T_e})
\end{equation}

\noindent
よって両辺の対数を取ると,

\begin{equation}
    log I_e = \frac{e}{k_B T_e} V_p + C
\end{equation}

\noindent
となる. したがって, $log I_e$ - $V_p$グラフにおける傾き($a$とする)の逆数を取ることで, 以下の%
式(8)によって$Te [eV]$が求まる.

\begin{equation}
    \frac{k_B T_e}{e} = \frac{1}{a}
\end{equation}

\noindent
$log I_e$と$V_p$をプロットしたグラフ以下に図2-2として示す.

\begin{figure}[H]
    \centering
    \captionsetup{labelformat=empty}
    \includegraphics[width=10cm]{Pictures/A5-2-3.png}
    \caption{図2-2: プローブ電流(対数表示)-電圧特性}
\end{figure}

\noindent
最小二乗法によって式(7)の線形近似を行うと, $a = 0.177$であり, %
式(8)に代入すると, $T_e = 5.65 [eV]$となる. また, %
式(5)より, $n_e = 1.74 \times 10^{11} [m^{-1}]$となる.

\section{考察}

\subsection{プローブ実験における抵抗の必要性}
 プローブ電圧がプラズマ電位$V_s$付近またはそれを超えた電子飽和電流領域に達すると, %
プローブには大量の電子飽和電流が流れ込もうとする. プラズマ密度が高い場合, %
この電流が非常に大きくなることでプローブ電源や計測機器に過負荷をかけ, 故障や損傷を%
引き起こす可能性がある. つまり, 本実験における抵抗の役割は, 電子飽和電流を制限する保護抵抗として機能し, %
回路の安全性を確保することである.

\subsection{考察課題(4)}
 以下に核融合, ヘリコンプラズマ, マグネトロンプラズマ, 太陽コロナ, 星間雲プラズマの%
電子密度と電子温度の領域を表3-1として示す.

\begin{table}[H]
    \centering
    \captionsetup{labelformat=empty}
    \caption{表3-1: 各種プラズマの電子密度と電子温度の代表的な領域}
    \begin{tabular}{|c|c|c|}
        \hline
        プラズマの種類 & 電子密度 $n_e [m^{-3}]$ & 電子温度 $T_e [eV]$ \\
        \hline
        核融合プラズマ & $10^{19} \sim 10^{21}$ & $10^3 \sim 3\times 10^4$  \\
        ヘリコンプラズマ & $10^{16} \sim 10^{19}$ & $2 \sim 8$  \\
        マグネトロンプラズマ & $10^{15} \sim 10^{17}$ & $1 \sim 5$ \\
        太陽コロナ & $10^{11} \sim 10^{15}$ & $10^2 \sim 10^3$  \\
        星間雲プラズマ & $10^{6} \sim  10^{9}$ & $10^{-2} \sim 10^0$  \\
        \hline
    \end{tabular}
\end{table}

\noindent
また, 以下に図4-1として, 各種プラズマの電子密度と電子温度の代表的な領域と実験値の比較したものを%
両対数グラフに領域を示す.

\begin{figure}[H]
    \centering
    \captionsetup{labelformat=empty}
    \includegraphics[width=10cm]{Pictures/A5-2-4.png}
    \caption{図4-1: 各種プラズマの電子密度と電子温度の代表的な領域と実験値の比較}
\end{figure}

\subsection{考察課題(6)}
 電子の速度分布関数 $f_e(\mathbf{v})$ がマクスウェル速度分布であるとき, その物理量を求める.

\begin{itemize}
    \item $n_e$: 電子密度
    \item $m_e$: 電子の質量
    \item $k_B$: ボルツマン定数
    \item $T_e$: 電子温度
\end{itemize}

\noindent
とする. 速度分布関数は以下の式(9)で与えられる.

\begin{equation}
    f_e(\mathbf{v_e}) = n_e \left(\frac{m_e}{2\pi k_B T_e}\right)^{1/2} \exp\left(-\frac{m_e v_e^2}{2 k_B T_e}\right)
\end{equation}

\noindent
 電子の平均速度 $\langle \mathbf{v_e} \rangle$は, 上の式(9)を全速度空間で積分することで求められる.%
マクスウェル分布は速度空間 ($\mathbf{v_e}$) において原点対称であるため、特定の方向に進む電子と反対方向に%
進む電子の数が等しく, 平均速度はゼロとなる. よって, 

\begin{equation}
    \langle \mathbf{v_e} \rangle = \frac{1}{n_e} \int_{-\infty}^{\infty} \mathbf{v} f_e(\mathbf{v}) d\mathbf{v} = 0
\end{equation}

\noindent
 電子の平均エネルギー $E = \frac{1}{2} m_e v^2$ は, 速度分布関数を用いて以下の式(11)で定義される.

\begin{equation}
    \langle E \rangle = \frac{\int_{-\infty}^{\infty} \left(\frac{1}{2} m_e v^2\right) f_e(\mathbf{v}) d\mathbf{v}}{\int_{-\infty}^{\infty} f_e(\mathbf{v}) d\mathbf{v}} 
\end{equation}

\noindent
ここで, $v_{th} = (2k_B T_e / m_e)^{1/2}$, $y = v / v_{th}$%
とすると, 式(11)は以下の式(12)のように書ける.

\begin{equation}
    \langle E \rangle = \frac{\frac{1}{2}m_e v_{th}^2 \int_{-\infty}^{\infty} [exp(-y^2)]y^2 d\mathbf{v}}{\int_{-\infty}^{\infty} exp(-y^2) d\mathbf{v}} 
\end{equation}

\noindent
この積分の結果は, 以下の式(13)により表される.

\begin{equation}
    \langle E \rangle = \frac{1}{2} k_B T_e
\end{equation}

\noindent
 最大確率速度について, 速度の大きさの分布関数$F(v)$を求めるために,%
まず, 速度分布関数$f_e (v)$を速度空間の球殻にわたって積分する.

\begin{equation}
    F(v) = 4\pi v^2 f_e(\mathbf{v}) = 4\pi n_e \left(\frac{m_e}{2\pi k_B T_e}\right)^{3/2} v^2 \exp\left(-\frac{m_e v^2}{2 k_B T_e}\right)
\end{equation}

\noindent
ただし, $f_e (v)$は3次元のものを使用した. $F(v)$ が最大となる $v$ は、$\frac{dF(v)}{dv} = 0$ を解くことで得られる.
以下の式(15)に結果を示す.

\begin{equation}
    v_{mp} = \sqrt{\frac{2 k_B T_e}{m_e}}
\end{equation}

\section{参考文献}
プラズマ物理学入門 丸善出版

\end{document}