% !TeX root = A5.tex
\documentclass{article}
\usepackage[utf8]{inputenc}
\usepackage{amsmath}  
\usepackage[dvipdfmx]{graphicx}
\usepackage{caption}
\usepackage{float}
\usepackage{url}
\usepackage[dvipdfmx, x11names]{xcolor}

\begin{document}
\section{検討課題}
\paragraph{(1)}

以下の図1-1にCGS電磁単位系とMKS単位系の換算表を示す.

\begin{table}[H]
    \centering
    \caption{MKSA単位系とCGS電磁単位系の比較表}
    \begin{tabular}{|l|l|l|}
    \hline
    \textbf{電気・磁気の量} & \textbf{MKSA単位} & \textbf{CGS電磁単位} \\
    \hline
    起電力・電位 & 1V (volt) & \(10^8\) emu \\
    電磁界の強さ & 1V/m (ampere) & \(10^6\) emu \\
    電流 & 1A (ampare) & \(10^{-1}\) Bi (biot) \\
    電流密度 & 1A/m² & \(10^{-5}\) emu \\
    抵抗 & 1Ω (ohm) & \(10^{9}\) emu \\
    抵抗率 & 1Ω・m & \(10^{11}\) emu \\
    コンダクタンス & \(1\Omega ^{-1}\) & \(10^{-9}\) emu \\
    電気量 & 1C (coulomb) & \(10^{-1}\) emu \\
    誘電束 & 1C & \(4 \pi /10\) emu \\
    誘電束密度 & 1C/m² & \(4 \pi 10^{5}\) emu \\
    静電容量 & 1F (farad) & \(10^{-9}\) emu \\
    誘電率 & 1F/m & \(4 \pi /10^{11}\) emu \\
    起磁力・磁位 & 1A & \(4 \pi / 10\) Gb (gilbert) \\
    磁界の強さ & 1A/m & \(4\pi \times 10^{3}\) (oersted) \\
    磁束 & 1Wb (weber) & \(10^{8}\) Mx (maxwell) \\
    磁束密度 & 1T (tesla) = 1 wb/m² & \(10^4\) G (gauss) \\
    磁極の強さ & 1Wb & \(10^{8} / 4\pi\) emu \\
    磁化の強さ & 1T & \(10^4\) emu \\
    インダクタンス & 1H & \(10^9\) emu \\
    磁気抵抗 & 1H (henry) & \(4\pi 10^9\) emu \\
    誘電率 & 1H/m & \(10^7 / 4\pi\) emu \\
    磁化率 & 1H/m & \(10^7 / (4\pi)^2\) emu \\
    \hline
    \end{tabular}
    \end{table}
    

\paragraph{(2)}
実験によって得られたデータを以下に図2-1として示す.

\begin{figure}[H]
    \centering
    \captionsetup{labelformat=empty}
    \includegraphics[width=10cm]{Pictures/Picture2-1.png}
    \caption{図2-1: 磁束計の較正}
\end{figure}

\paragraph{(3)}
以下に図3-1, 図3-2として, コイル(大)とコイル(小)の実験値を較正した%
正規化磁化曲線をそれぞれ示す.

\begin{figure}[H]
    \centering
    \captionsetup{labelformat=empty}
    \includegraphics[width=10cm]{Pictures/Picture3-1.png}
    \caption{図3-1: コイル(大)の較正した正規化磁化曲線}
\end{figure}

\begin{figure}[H]
    \centering
    \captionsetup{labelformat=empty}
    \includegraphics[width=10cm]{Pictures/Picture3-2.png}
    \caption{図3-2: コイル(小)の較正した正規化磁化曲線}
\end{figure}

また、以下に図3-3として, ヒステリシスループを示す.
\begin{figure}[H]
    \centering
    \captionsetup{labelformat=empty}
    \includegraphics[width=10cm]{Pictures/Picture3-1.png}
    \caption{図3-3: コイル(小)の較正した正規化磁化曲線}
\end{figure}


\paragraph{(4)}
図3-1より, コイル(大)の飽和磁束密度は, $0.7745 T$である.

また, 最大透磁率は切片の傾きの最大値から, $0.016$である.
真空の透磁率は$1.257 \times 10^{-6} mkgs^{-2}A^{-2}$とすると, 最大比透磁率は以下のようになる.

同様に, 図3-2より, コイル(小)の飽和磁束密度は,$1.541 T$である.
また, 最大透磁率は切片の傾きの最大値から, $0.066$

\paragraph{(5)}


\paragraph{(6)}
図3-3のヒステリシスループにおいて, $I = 0$での$|B|$の平均を求めると,
残留磁束密度$B_r$は以下の式(1) で表される

$B = 0$の時の$|H|$の平均を求めると, 
保磁力$H_c$は以下の式(2)で表される

また, ヒステリシス損は以下の通りに計算を行う。

\end{document}